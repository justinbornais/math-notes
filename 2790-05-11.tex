\documentclass{article}
\usepackage{color}
\usepackage{amsmath}
\usepackage{graphicx}
\usepackage{geometry}
\usepackage{xcolor}
\geometry{letterpaper, total={170mm,257mm}, left=20mm, top=20mm}
\graphicspath{ {./images/} }
\newtheorem{theorem}{Theorem}[section]
\newcommand{\Lim}[1]{\raisebox{0.5ex}{\scalebox{0.8}{$\displaystyle \lim_{#1}\;$}}}
\newcommand{\p}{\partial}
\newcommand{\dg}{^{\circ}}
\newcommand{\mat}[4]{\begin{bmatrix} #1 & #2 \\ #3 & #4 \end{bmatrix}}
\newcommand{\mathree}[9]{\begin{bmatrix} #1 & #2 & #3 \\ #4 & #5 & #6 \\ #7 & #8 & #9 \end{bmatrix}}

\title{Differential Equations}
\author{Justin Bornais}
\date{May 11, 2023}

\begin{document}
\maketitle
\section{Introduction to Differential Equations}
\paragraph{Differential Equation} Any equation that is differentiable. An example is $3y''+xy'-x^2=e^x$

The \textbf{order} of a DE is the highest order of derivative.
\begin{equation}
    \textbf{Linear DE}
    \left\{
        \begin{array}{lr}
            (i) y, y', y'', ..., & \text{Cannot have more than one power} \\
            (ii) $\text{We can have }$, & x^n, e^x, \sin{x},$\text{ etc. but not }$ y^n, e^y, \cos{y},$\text{ etc.}$
        \end{array}
    \right\}
\end{equation}

Example of nonlinear DE: $3x^3y'' + yy' = e^x$ \\ \\
$y=f(x)$ is a \textbf{solution} of a DE on an interval I (interval of existence of solution):
\begin{itemize}
    \itemsep 0em
    \item y satisfies the DE.
    \item $y$, $y'$, $y''$, ... are continuous on I.
\end{itemize}

\subsection{Initial Value Problems (IVP)}
The first-order DE $y'=f(x,y)$ subject to $y(x_0)=y_0$ $\rightarrow$ is an IVP. \\
The second-order DE $y''=f(x,y,y')$ subject to $\left(y(x_0)=y_0, y'(x_0)=y_1\right)$ is an IVP, while $\left(y(x_0)=y_0, y(x_1)=y_1\right)$ is a \textbf{Boundary Value Problem (BVP)}.
\begin{itemize}
    \itemsep 0em
    \item Note: After $y_1$ both cases can be used.
\end{itemize}

\subsubsection{Example 1}
$y=c_1\cos{x}+c_2\sin{x}$ is a soluation of the DE $y''+y=0$. Find the solution subject to the conditions $y(\pi)=1, y'(\pi)=-2$

\paragraph{Solution:} $y(\pi)=1 \Rightarrow 1=c_1\cos{\pi}+c_2\sin{\pi}$

$\Rightarrow 1=-c_1 \Rightarrow c_1=-1$ \\\\
Then: $y=c_1\cos{x}+c_2\sin{x} \Rightarrow y'=-c_1\sin{x}+c_2\cos{x}$

Using $y'(\pi)=-2$ $\Rightarrow -2=-c_1\sin{pi}+c_2\cos{pi} \Rightarrow -2=-c_2 \Rightarrow c_2=2$\\\\
The solution is $y=-\cos{x}+3\sin{x}$

\subsubsection{Example 2}
$y=\frac{1}{x^2+c}$ is the one parameter solution of the DE $y'+2xy^2=0$. Find a solution of the IVP:\\ $y'+2xy^2=0, y(-3)=\frac{1}{5}$. Give the longest interval over which the solution is defined.

\newpage \paragraph{Solution:} Using $y(-3)=\frac{1}{5}$, $y=\frac{1}{x^2+c}\Rightarrow \frac{1}{5}=\frac{1}{(-3)^2+c}$

$\Rightarrow \frac{1}{5}=\frac{1}{9+c}$

$\Rightarrow 9+c=5$

$\Rightarrow c=-4$ \\
Thus $y=\frac{1}{x^2-4}$ is a solution of the IVP. \\\\
$y$ is continuous when $x^2-4\ne 0 \Rightarrow x^2\ne 4 \Rightarrow x\ne \pm 2$

$y=\frac{1}{x^2-4}=\left(x^2-4\right)^{-1}$

$y'=-1\left(x^2-4\right)^{-2}(2x)=\frac{-2x}{(x^2-4)^2}$ is continuous when $x^2-4\ne0\Rightarrow x\ne\pm2$ \\\\
The longest interval is $\left(-\infty, -2\right)$ on which the solution is defined.\\

\subsubsection{Not every DE is solvable}
Consider the first-order IVP $xy'=2y,y(0)=0$. $y=0$ is a solution $y=0,y'=0 \Rightarrow x(0)=2(0) \Rightarrow 0=0$
\begin{itemize}
    \itemsep 0em
    \item \textbf{Note:} a solution must be valid for all values of $x$.
\end{itemize}
$y=x^2$ is also a solution $\Rightarrow y'=2x \Rightarrow x(2x)=2x^2\Rightarrow 2x^2=2x^2\Rightarrow y(0)=0$

\subsection{Existence Theorem}
\begin{theorem}[1.2.1]
    Let $R$ be a rectangular region $R=\left\{ (x,y) | a\leq x\leq b, c\leq y\leq d\right\}$ that contains $(x_0,y_0)$ in its interior.
    If f and $\frac{df}{dy}$ are continous on R, then there exists an interval $(x_0-h,x_0+h)$ in R on which the IVP
    $y'=f(x,y), y(x_0)=y_0$ has a unique solution.
\end{theorem}
\begin{itemize}
    \item \textbf{Note: } We need to have the form $y'=f(x,y)$ to decide $f$.
\end{itemize}
Examples:
\begin{itemize}
    \item $f(x)=x^3+\cos{x}\Rightarrow f'(x)=3x^2-\sin{x}$
    \item $f(x,y)=x^3\cos{y}+e^y-x^7$
\end{itemize}
For a partial derivative of $f$ with respect to $x\rightarrow\frac{df}{dx}$, we treat $y$ as a constant. \\
For a partial derivative of $f$ with respect to $y\rightarrow\frac{df}{dy}$, we treat $x$ as a constant. \\
An example: $f(x,y)=x^3\cos{y}+e^y-x^7$
\begin{itemize}
    \item $\frac{df}{dx}=(\cos{y})(3x^2)+0-7x^6$
    \item $\frac{df}{dy}=x^3(-\sin{y})+e^y+0$
\end{itemize}

\subsubsection{Example 1}
Determine whether the existence theorem guarantees that the IVP $xy'=2y, y(0)=0$ has a unique solution.

\paragraph{Solution:} $y'=\frac{2y}{x}\rightarrow f(x,y)=\frac{2y}{x}$ is continuous when $x\ne0$
\\Conditions of existence theorem are not satisfied. So there is no guarantee of a unique solution.

\subsubsection{Example 2}
Determine a region R of the xy-plane for which the DE $\left(1+y^3\right)y'=x^2$ would have a unique solution in an interval around (0,2) (a unique solution passing through (0,2)).

\paragraph{Solution:} $(1+y^3)y'=x^2\Rightarrow y'=\frac{x^2}{1+y^3}\Rightarrow f(x,y)=\frac{x^2}{1+y^3}=x^2(1+y^3)^{-1}$ is continous when\\$1+y^3\ne0, y^3\ne1, y\ne-1$
\\\\$\frac{df}{dy}=x^2\left[ -(1+y^3)^{-2}(3y^2) \right] =\frac{-3x^2y^2}{(1+y^3)^2}$
\\$\frac{df}{dy}$ is continuous when $1+y^3\ne0\Rightarrow y\ne-1$

Define $R=\left\{(x,y)|-10\leq x \leq 10, 0\leq y \leq 9\right\}$
\begin{itemize}
    \item \textbf{Note: } the boundaries of $x$ can be anything since there are no restrictions on $x$, so long as it contains $x=0$.
    \item The boundaries of $y$ must contain $y=2$ and must not cross $y=-1$.
\end{itemize}
\newpage
\subsubsection{Example 3}
Find a region in the xy-plane on which the IVF $(1+y^3)y'=x^2, y(1)=-3$ would have a unique solution.

\paragraph{Solution:}
From example 2, $f,\frac{df}{dy}$ are continous when $y\ne-1$
\\\\$R=\left\{(x,y)|-3\leq x\leq 5, -6\leq y\leq -2\right\}$ where the x boundaries contain $x=1$ and the y boundaries contain $y=-3$.

\newpage
\section{Solutions of First Order Differential Equations}
In last class, 2.2, separable equation $\frac{dy}{dx}=g(x)h(y)$ which can turn into $\frac{dy}{h(y)}=g(x)dx$ and this can be integrated.
\\When integrating, it must occur with the same variable.
\begin{itemize}
    \item $\int x^2dx=\frac{x^3}{3}$
    \item $\int x^2du$ cannot integrate.
\end{itemize}
There are singular and implicit solutions.

\subsection{Linear Equations}
From section 1.1, nth order linear DE is represented by $\frac{d^ny}{dx^n}$ or $y^{(n)}$
\\The equation is $a_n(x)\frac{d^ny}{dx^n}+a_{n-1}y^{(n-1)}+\dots+a_2(x)y''+a_1(x)y'+a_0(x)y=g(x)$
\\\textbf{First order} is only concerned with $a_1(x)y'+a_0(x)y=g(x)$
\\Linear equations are easy to solve and get explicit solutions easily. However there are \textbf{no} singular solutions of linear DEs.

\subsubsection{Solving Linear 1st Order DEs}
\paragraph{Step 1} Wrote tje DE in the standard form $y'+P(x)y=f(x)$\\
$a_1(x)y'+a_0(x)y=g(x)$
$y'+\frac{a_0(x)}{a_1(x)}y=\frac{g(x)}{a_1(x)}\Rightarrow y'+P(x)y=f(x)$ (standard form of 1st order linear DE)

\paragraph{Step 2} Find the integrating factor (I.F.)
\\I.F.$=e^{\int\!P(x)dx}$ (do not write $+C$ in this integration since that adds a wasted simplification step)

\paragraph{Step 3} Multiply every term of the standard form equation (from \textbf{Step 1}) by the I.F.
\large \\$e^{\int\!P(x)dx}y'+e^{\int\!P(x)dx}P(x)y=f(x)e^{\int\!P(x)dx}$
\normalsize
\\The LHS will automatically be $\frac{d}{dx}((I.F.)y)$ i.e. $\frac{d}{dx}\left(e^{\int\!P(x)dx}y\right)$
\\$=e^{\int\!P(x)dx}y'+ye^{\int\!P(x)dx}(P(x))$

\paragraph{Step 4} Integrate both sides w.r.t. $x$.
\\$(I.F.)y=\int f(x)e^{\int P(x)dx}dx$
\\The explicit solution will be \Large $y=\frac{\int f(x)e^{\int P(x)dx}dx}{I.F.}$\normalsize
\\\\We get $y'+P(x)y=f(x)$. The interval of existence of solution is the interval over which \textbf{$P(x)$ and $f(x)$ are continuous}.
\\As a reminder of section 1.2, for any 1st order DE $y'=f(x,y);y(x_0)=y_0$, if $f\&\frac{df}{dy}$ is continuous on R, then a unique solution exists.
\\For linear equations, $y'=f(x)-P(x)y$, $\frac{df}{dy}=0-P(x)(1)$, where $f(x)-P(x)y=f(x,y)$

\subsubsection{Example 1}
Solve the differential equation $y'-2xy=x$. Give the largest interval I over which the solution is defined.

\paragraph{Solution} $y'-2xy=x$ is a linear equation where $P(x)=-2x$, $f(x)=x$
\\Since $P(x)$ and $f(x)$ are continuous on $R$, then the largest interval of existence of the solutio is $I=(-\infty,\infty)$
\\\\$\int P(x)dx=\int -2xdx=\frac{-2x^2}{2}=-x^2$ (remember, don't write $+C$)
\\Thus $I.F.=e^{\int\!P(x)dx}=e^{-x^2}$
\\\\Next, multiple the DE by the I.F.
\\$e^{-x^2}y'-2xe^{-x^2}y=e^{-x^2}x$, $\frac{d}{dx}\left(e^{-x^2}y\right)=e^{-x^2}x$
\\Integrating w.r.t. $x$: $e^{-x^2}y=\int e^{-x^2}xdx$, let $u=-x^2$, $du=-2xdx$
\\$\int e^u\frac{du}{-2}=\frac{-1}{2}\int e^udu=\frac{-1}{2}e^u+C=\frac{-e^{-x^2}}{2}+C$
\\Thus $e^{-x^2}y=\frac{-1}{2}e^{-x^2}+C\Rightarrow y=\frac{\frac{-1}{2}e^{-x^2}}{e^{-x^2}}+\frac{C}{e^{-x^2}}\Rightarrow y=\frac{-1}{2}+Ce^{x^2}$
\newpage
\paragraph{Question:} Is the equation $y'-2xy=x$ separable?\\
\textbf{Answer:} If you can separate $x$ from $y$, then it is separable. So $y'-2xy=x\Rightarrow y'=2xy+x\Rightarrow y'=x(2y+1)$
Thus, the equation is separable.


\subsubsection{Example 2}
Solve the IVP: $xy'+2y=12x^4;\:y(1)=4$. Give the largest interval over which the solution is defined.
\paragraph{Solution} $xy'+2y=12x^4\Rightarrow y'+\frac{2}{x}y=12x^3$ is standard form. $P(x)=\frac{2}{x}\;\&\;f(x)=12x^3$ are continuous on R except at $x=0$. Thus the longest interval of existence of the solution is $(0, \infty)$.
\\$I.F.=e^{\int\!P(x)dx}=e^{\int\frac{2}{x}dx}=e^{2\ln|x|}=e^{\ln{|x|}^2}=|x|^2$
\\\\Next we multiply the standard form function by the I.F. to get $x^2y'+x^2\left(\frac{2}{x}\right)y=x^2(12x^3)\Rightarrow \frac{d}{dx}(x^2y)=12x^5$
\\Now we integrate w.r.t. $x$: $x^2y=\int 12x^5dx=2x^6+C$
\\Thus $y=\frac{2x^6}{x^2}+\frac{C}{x^2}\Rightarrow y=2x^4+\frac{C}{x^2}$
\\\\For the IVP, $y(1)=4$, $4=2(1)+\frac{C}{1}\Rightarrow C=2$
\\Thus the solution of the IVP is $y=2x^4+2$
\\Now, find the transient term in the solution: $\lim_{x \to \infty}\frac{2}{x^2}=0\Rightarrow\frac{2}{x^2}$ is the transient term.
\begin{itemize}
    \item \textbf{Note:} a \textbf{transient term} is a term which approaches 0 as $x$ approaches $\infty$.
\end{itemize}

\subsubsection{Example 3}
Find the general solution (a linear DE) of $xy'+(1+x)y=e^{-x}\sin^2x$ (or, solve the DE).

\paragraph{Solution} $y'+\frac{1+x}{x}y=\frac{e^{-x}\sin^2x}{x}$ is standard form, $P(x)=\frac{1+x}{x}$
\\$\int\!P(x)dx=\int\frac{1+x}{x}dx=\int\left(\frac{1}{x}+1\right)dx=\ln{|x|}+x$ (no $+C$)
\\Thus $I.F.=e^{\int\!P(x)dx}=e^{\int \ln{|x|}+xdx}=e^{\ln{|x|}}\cdot e^x=|x|e^x\Rightarrow I.F.=xe^x$
\\$xe^xy'+xe^x(1+x)y=xe^x\frac{e^{-x}\sin^2x}{x}\Rightarrow\frac{d}{dx}(xe^xy)=\sin^2x$
\\\\Integrating w.r.t. $x$: $xe^xy=\int\sin^2xdx=\frac{1}{2}\int (1-\cos 2x)dx=\frac{1}{2}\left(x-\frac{\sin 2x}{2}\right)$
\\Thus $xe^xy=\frac{1}{2}\left(x-\frac{sin2x}{2}\right)+C\Rightarrow y=\frac{\frac{x}{2}-\frac{\sin 2x}{4}}{xe^x}+\frac{C}{xe^x}$

\newpage
\subsection{(2.4) Exact Equations}
First order DEs:
\begin{equation}
    \left\{
        \begin{array}{lr}
            2.2 \rightarrow\;\text{Separable} & \frac{dy}{dx}=g(x)h(y)\\
            2.3 \rightarrow\;\text{Linear equation} & y'+P(x)y=f(x)\\
        \end{array}
    \right\}
\end{equation}
\\Differential: $\delta x=dx$
\begin{itemize}
    \itemsep 0em
    \item $y=f(x)\rightarrow$ differential is $dy=f'dx$
    \item $z=f(x,y)\rightarrow$ differential is $dz=\frac{df}{dx}+\frac{df}{dy}dy$
    \item $\frac{df}{dx}\rightarrow$ treat $y$ as a constant.
    \item $\frac{df}{dx}\rightarrow$ treat $x$ as a constant.
\end{itemize}

Consider $f(x,y)=x^3+x^2y\Rightarrow \frac{df}{dx}=3x^2+(2x)y$ and $\frac{df}{dy}=0+x^2(1)$
\\$x^3+x^2y=7\;[1]$, take derivatives to get: $(3x^2+2xy)dx+x^2dy=0 \;[2]$. [1] is a solution of the DE [2].
\paragraph{}A DE of the form $\frac{df}{dx}dx+\frac{df}{dy}dy=0$ is called an \textbf{exact DE}.
\paragraph{How to decide if a DE is exact}A DE of the form $M(x,y)dx+N(x,y)dy=0$ is an exact equation iff $\frac{dM}{dy}=\frac{dN}{dx}$ or $M_y=N_x$
\paragraph{How to solve the exact equation}We have two equations $\frac{df}{dx}=M\;[1]$ and $\frac{df}{dy}=N\;[2]$
\begin{itemize}
    \itemsep 0em
    \item The partial integration $\int\frac{df}{dx}dx=f(x,y)$
    \item The partial integration $\int\frac{df}{dy}dy=f(x,y)$
\end{itemize}
Doing the partial integration of [1]: $f(x,y)=\int Mdx+g(y)$, where $g(y)$ is the constant of integration.
\\Find $\frac{df}{dy}$ and substitute in [2] to find $g(y)$.
\\$f(x,y)=...$ (the solution is $f(x,y)=C$)

\subsubsection{Exampole 1}
Determine whether the DE is exact or not. If it is exact, then solve it. $y'=\frac{-2xy}{1+x^2}$

\paragraph{Solution} $y'=\frac{-2xy}{1+x^2}\Rightarrow\frac{dy}{dx}=\frac{-2xy}{1+x^2}\Rightarrow (1+x^2)dy=-2xydx\Rightarrow2xydx+(1+x^2)dy=0$
\\$M_y=\frac{dM}{dy}=2x(1)$ and $N_x=\frac{dN}{dx}=2x$, $M_y=N_x\Rightarrow$ exact.
\\\\\textbf{To solve}, $\frac{df}{dx}=M$ and $\frac{df}{dy}=N$
$\Rightarrow\frac{df}{dx}=2xy\;[1]$ and $\frac{df}{dy}=1+x^2\;[2]$
\\Perform partial integration of [1] w.r.t. $x$:
$f(x,y)=2y\frac{x^2}{2}+g(y)\;[3]\Rightarrow \frac{df}{dy}=(1)x^2+g'(y)$
\\Next substitute equation [2]:
$x^2+g'(y)=1+x^2\Rightarrow g'(y)=1$ (cannot have any $x$ terms)
\\Integrating w.r.t. $y$: $g(y)=\int1dy\Rightarrow g(y)=y$ or $g(y)=y+C$ (preferred without the $+C$)
\\Substitute $g(y)$ in [3] to get $f(x,y)=x^2y+y$ or $f(x,y)=x^2y+y+C$
\\\\The solution of the DE is $f(x,y)=C\Rightarrow x^2y+y=C$ or $f(x,y)=0\Rightarrow x^2y+y+C=0$

\subsubsection{Example 2}
Determine whether the DE is exact or not. If it is exact, then solve it. $(xy+y^2)dx+(x^2+xy)dy=0$

\paragraph{Solution} $M_y=x(1)+2y$ and $N_x=2x+(1)y \Rightarrow M_x\ne N_y\Rightarrow$ not exact.

\subsubsection{Example 3}
Determine whether the DE is exact or not. If it is exact, then solve it. $(x+\sin{y})dx=(e^y=x\cos{y})dy$

\newpage\paragraph{Solution} $(x+\sin{y})dx-(e^y-x\cos{y})dy=0$
\\Here, $M=x+\sin{y}$ and $N=-(e^y-x\cos{y})=-e^y+x\cos{y}$
\\$M_y=\cos{y}$ and $N_x=(1)\cos{y}\Rightarrow M_y=N_x\Rightarrow$ Exact.
\\\\\textbf{To solve}, $\frac{df}{dx}=M\Rightarrow\frac{df}{dx}=x+\sin{y}\;[1]$
\\$\frac{df}{dy}=N\Rightarrow\frac{df}{dy}=-e^y+x\cos{y}\;[2]$
\\Partial integration of [2] w.r.t. $y$: $f(x,y)=-e^y+x\sin{y}+g(x)\;[3]$
\\$\frac{df}{dx}=0+(1)\sin{y}+g'(x)$
\\Substitute in [1]: $\sin{y}+g'(x)=x+\sin{y}\Rightarrow g'(x)=x$ (cannot have $y$) $\Rightarrow g(x)=\frac{x^2}{2}$
\\Substitute $g(x)$ in [3]: $f(x,y)=-e^y+x\sin{y}+\frac{x^2}{2}$
\\The solution is $f(x,y)=C\Rightarrow -e^y+x\sin{y}+\frac{x^2}{2}=C$

\subsubsection{Example 4}
Solve the IVP: $(ey-xe^{xy})y'=2+ye^{xy}\;;\;y(0)=1$

\paragraph{Solution} $(2y-xe^{xy})dy=(2+ye^{xy})dx\Rightarrow(2+ye^{xy})dx-(2y-xe^{xy})=0$
\\$M=2+ye^{xy}\Rightarrow\frac{dM}{dy}=0+(1)e^{xy}+ye^{xy}(x)$
\\$N=-2y+xe^{xy}\Rightarrow\frac{dN}{dx}=0+(1)e^{xy}+xe^{xy}(y)$
\\Since $M_y=N_x$, this is an exact equation.
\\\\$\frac{df}{dx}=M\Rightarrow\frac{df}{dx}=2+ye^{xy}\;[1]$
\\$\frac{df}{dy}=N\Rightarrow\frac{df}{dy}=-2y+xe^{xy}\;[2]$
\\Partial integration of [1] w.r.t. x: $f(x,y)=\int(2+ye^{xy})dx=2x+y\frac{e^{xy}}{y}+g(y)\;[3]$
\\$\frac{df}{dy}=0+e^{xy}(x)+g'(y)$
\\Substitute in [2]: $xe^{xy}+g'(y)=-2y+xe^{xy}\Rightarrow g'(y)=-2y$
\\$g(y)=\int -2ydy=-2\frac{y^2}{2}$
\\Substituting in [3]: $f(x,y)=2x+e^{xy}-y^2$
\\The solution of the DE is $f(x,y)=C\Rightarrow2x+e^{xy}-y^2=C$
\\\\$y(0)=1\Rightarrow\text{Substitute }x=0\text{ and }y=1:\;2(0)+e^{(0)(1)}-(1)^2=C\Rightarrow1-1=C\Rightarrow C=0$
\\The solution of the IVP is $ex+e^{xy}-y^2=0$

\subsection{More on Exact Equations}
Sometimes, we can multiply the DE by an integrating factor and the DE becomes an exact DE. We solve by the mothod of exact equations:
\begin{enumerate}
    \itemsep 0em
    \item If $\frac{M_y-N_x}{N}$ is a function of $x$ only (no $y$ terms), then $I.F.=\mu=e^{\int\frac{M_y-N_x}{N}dx}$
    \item If $\frac{N_x-M_y}{M}$ is a function of $y$ only (no $x$ terms), then $I.F.=\mu=e^{\int\frac{N_x-M_y}{M}dy}$
\end{enumerate}

\subsubsection{Example 1}
Find an I.F. to make this DE exact and then solve it: $(y^2-y)dx+xdy=0$
\paragraph{Solution} $M=y^2-y,\;M_y=2y-1,\;N=x,\;N_x=1$\qquad $M_y\neq N_x\Rightarrow$ not exact.
\\$\frac{M_y-N_x}{N}=\frac{(2y-1)-1}{x}$ (not in terms of $x$ only)
\\$\frac{N_x-M_y}{M}=\frac{1-(2y-1)}{y^2-y}=\frac{1-2y+1}{y^2-y}=\frac{2-2y}{y^2-y}=\frac{-2(1-y)}{y(y-1)}=\frac{-2}{y}$ (good!)
\\$I.F.=\mu=e^{\int\frac{N_x-M_y}{M}dy}=e^{\int\frac{-2}{y}dy}=e^{-2\ln|y|}=e^{\ln|y|^{-2}}=|y|^{-2}=\frac{1}{y^2}$
\\\\Multiply the given DE by $\frac{1}{y^2}$: $\frac{1}{y^2}(y^2-y)dx+\frac{1}{y^2}xdy=0\Rightarrow(1-\frac{1}{y})dx+\frac{x}{y^2}dy=0$
\\Now, $M=1-\frac{1}{y}$ and $N=\frac{x}{y^2}\Rightarrow M_y=-(-1y^{-2})=\frac{1}{y^2},\;\;N_x=\frac{1}{y^2}(1)=\frac{1}{y^2}$ (unnecessary to check)
\newpage To solve: $\frac{df}{dx}=M\Rightarrow\frac{df}{dx}=1-\frac{1}{y}\;[1]$
\qquad $\frac{df}{dy}=N\Rightarrow\frac{df}{dy}=\frac{x}{y^2}\;[2]$
\\Partial integration of [1] w.r.t. $x\Rightarrow f(x,y)=x-\frac{1}{y}(x)+g(y)\;[3]$
\\$\frac{df}{dy}=0+\frac{x}{y^2}+g'(y)$ (substitute this into [2])
\\$\frac{x}{y^2}+g'(y)=\frac{x}{y^2}\Rightarrow g'(y)=0\Rightarrow g(y)=0$ or $g(y)=C$
\\So [3] gives $f(x,y)=x-\frac{x}{y}$ or $f(x,y)=x-\frac{x}{y}+C$
\\Thus the solution is $f(x,y)=C\Rightarrow x-\frac{x}{y}=C$ or $f(x,y)=0\Rightarrow x-\frac{x}{y}+C=0$
\\\\Now, this is only the case if $y\neq0$ (since the I.F. involves $y$ being in the denominator).
\\So we can check if $y=0$ is a solution: $dy=y'dx=0dx=0$
\\$(0^2-0)dx+xdy=0\Rightarrow(0^2-0)dx+x(0)=0\Rightarrow0+0=0\Rightarrow y=0$ is a solution.

\subsubsection{Example 2}
Find an I.F. to make this DE exact: $y(x+y+1)dx+(x+2y)dy=0$ (do not solve the equation)
\paragraph{Solution}
$M=xy+y^2+y$ and $N=x+2y$\qquad $M_y=x+2y+1$ and $N_x=1$
\\$\frac{M_y-N_x}{N}=\frac{(x+2y+1)-1}{x+2y}=\frac{x+2y}{x+2y}=1\Rightarrow I.F.=\mu=e^{\int 1dx}=e^x$

\subsection{Solutions by Substitution}
We do 3 kinds of substitutions which can be used to solve 1st order DEs.
\subsection{Homogeneous Equations}
A function $f(x,y)$ is called a homogeneous function of degree $a$ if $f(tx,ty)=t^af(x,y)$
\\\\To determine if the equation $f(x,y)=x^2+xy$ is homogeneous, calculate $f(tx,ty)$.
\\$f(tx,ty)=(tx)^2+(tx)(ty)=t^2x^2+t^2xy=t^2(x^2+xy)=t^2f(x,y)\Rightarrow$ Thus $f$ is a homogeneous function of degree 2.
\\For more functions: \begin{itemize}
    \itemsep 0em
    \item $f(x,y)=x+3$, $f(tx,ty)=tx+3\Rightarrow$ not homogeneous.
    \item $f(x,y)=\sin{x}-\sin{y}$\quad$f(tx,ty)=\sin{tx}-\sin{ty}\Rightarrow$ not homogeneous. (Usually anything with $\sin{x}$ or $\sin{y}$ is not homogeneous)
    \item $f(x,y)=\sin{\frac{x}{y}}$\quad$f(tx,ty)=\sin{\frac{tx}{ty}}=\sin{\frac{x}{y}}=t^0\sin\frac{x}{y}\Rightarrow$ homogeneous. (this is an exception to the above)
    \item $f(x,y)=\ln{x}-\ln{y}$\quad$f(tx,ty)=\ln{tx}-\ln{ty}=\ln{\frac{tx}{ty}}=\ln{\frac{x}{y}}=t^0(\ln{x}-\ln{y})\Rightarrow$ homogeneous of degree 0.
\end{itemize}
A differential equation $Mdx+Ndy=0$ is homogeneous if both $M$ and $N$ are homogeneous functions of the same degree.
\paragraph{Example 1} For $(x^2+xy)dx+xy^2dy=0$, $M$ is homogeneous of degree 2 and $N$ is homogeneous of degree 3. Thus the equation is not homogeneous.
\paragraph{Example 2} For $(x+3)dx+ydy=0$, it is not homogeneous because $M$ is not homogeneous.
\paragraph{Example 3} For $(x^2+xy)dx+(xy-y^2)dy=0$, since both $M$ and $N$ are homogeneous of degree 2, thus the equation is homogeneous. (you can often check the power of $M$ and $N$ to check if it's homogeneous)

\subsubsection{Solving a Homogeneous Equation}
\textbf{Case 1:} Let $y=ux$ and $dy=udx+udu\to$ the homogeneous equation will become a separable equation in $u$ and $x$.
Solve it and then replace $u=\frac{y}{x}$ in the solution.
\\\textbf{Case 2:} Let $x=vy$ and $dx=vdy+ydv\to$ the homogeneous equation will become a separable equation in $v$ and $y$.
Solve it and then replace $v=\frac{x}{y}$ in the solution.

\newpage\paragraph{Example} Solve the DE by using an approprite substitution: $ydx-2(x+y)dy=0$
\paragraph{Solution} Let $y=ux$ and $dy=udx+xdu$\qquad\textbf{or} let $x=vy$ and $dx=vdy+ydv$
\\$uxdx-(2x+2ux)(udx+xdu)=0$\qquad\qquad\qquad\textbf{or} $y(vdy+ydv)-(2vy+2y)dy=0$
\\\textcolor{red}{This is harder to solve}\qquad\qquad\qquad\qquad\qquad\qquad\textbf{or} $vydy+y^2dv-2vydy-2ydy=0$
\\$y^2dv-vydy-2ydy=0$
\\$y^2dv=vydy+2ydy\Rightarrow y^2dv=ydy(v+2)\Rightarrow\frac{dv}{v+2}=\frac{ydy}{y^2}\Rightarrow\frac{dv}{v+2}=\frac{1}{y}dy\to$ a separable equation.
\\\\If we went the $u$ way: $\int\frac{-dx}{x}=\int\frac{2+2u}{u+2u^2}du\dots$ needs partial fraction.
\\Instead, $\int\frac{dv}{v+2}=\int\frac{1}{y}dy\Rightarrow \ln|v+2|=\ln|y|+C$
\\Replace $v$ by $\frac{x}{y}\Rightarrow\ln|\frac{x}{y}+2|=\ln|y|+C$
\\\\Is $y=0$ also a solution?
\\Well, $y=0\Rightarrow dy=0$. Substitute into original equation: $(0)dx-2(x+0)(0)=0\Rightarrow0-0=0\\\Rightarrow y=0$ is also a solution.
\\$v+2=0\Rightarrow\frac{x}{y}+2=0\Rightarrow\frac{x}{y}=-2\Rightarrow x=-2y\Rightarrow y=\frac{-x}{2}$
\\Is this a solution? Well, $y=\frac{-x}{2}\Rightarrow dy=y'dx=\frac{-1}{2}dx$
\\Substitute into original equation: $\frac{-x}{2}dx-2(x-\frac{x}{2})(\frac{-1}{2}dx)=0\Rightarrow
\frac{-x}{2}dx+\frac{x}{2}dx=0\Rightarrow0=0\\\Rightarrow y=\frac{-x}{2}$ is also a solution.

\subsection{Bernoulli's Equation}
$\frac{dy}{dx}+P(x)y=f(x)y^n$ where $n\in R, n\neq0$ and $n\neq1$
\\If $n=0\Rightarrow\frac{dy}{dx}+P(x)y=f(x)(1)\rightarrow$ linear.
\\If $n=1\Rightarrow\frac{dy}{dx}+P(x)y=f(x)y\Rightarrow\frac{dy}{dx}+(P(x)-f(x))y=0\rightarrow$ linear equation.
\\If $n\neq0$ and $n\neq1$ then we use the substitution $u=y^{1-n}$ and the Bernoulli's equation changes to a linear equation $\to$ solve linear equation $\to$ replace $u$.
\\\\Another way: for the equation $\frac{dy}{dx}+P(x)y=f(x)y^n$, let $u=y^{1-n}$ and the equation turns into:
\\$\frac{du}{dx}+(1-n)P(x)u=(1-n)f(x)\Rightarrow$ linear equation in $u$.

\subsubsection{Example 1}
Solve the DEs using an appropriate substitution: $\frac{dy}{dx}-y=e^xy^2$
\paragraph{Solution} This is Bernoulli's equation with $n=2$.
\\Let $u=y^{1-n}=y^{1-2}=y^{-1}=\frac{1}{y}$ if $y\neq0$
\\$\Rightarrow y=\frac{1}{u}=u^{-1}$ and $\frac{dy}{dx}=-1u^{-2}\frac{du}{dx}=\frac{-1}{u^2}\frac{du}{dx}$
\\Substituting in the original equation: $\frac{-1}{u^2}\frac{du}{dx}-\frac{1}{u}=e^x\frac{1}{u^2}$
\\Multiplying by $-u^2\Rightarrow\frac{du}{dx}+u=-e^x\;[1]$
\\\\An alternate route: Let $u=y^{1-n}$. Then Bernoulli's equation becomes $\frac{du}{dx}+(1-n)P(x)u=(1-n)f(x)$
\\$\Rightarrow\frac{du}{dx}-(-1)u=(-1)e^x$
\\$\frac{du}{dx}+u=-e^x$ (linear in $u$)
\\\\$I.F.=e^{\int P(x)dx}=e^{\int 1dx}=e^x$
\\Multiplying [1] by $e^x\Rightarrow e^x\frac{du}{dx}+e^xu=-e^xe^x$
\\$\frac{d}{dx}(e^xu)=-e^{2x}$
\\Integrating w.r.t. $x\Rightarrow e^xu=\int -e^{2x}dx\Rightarrow e^xu=\frac{-e^{2x}}{2}+C\Rightarrow u=\frac{-e^{2x}}{2e^x}+\frac{C}{e^x}$
\\Replace $u=\frac{1}{y}\Rightarrow\frac{1}{y}=-\frac{e^x}{2}+Ce^{-x}$

\subsubsection{Example 2}
Solve the DEs using an appropriate substitution: $\frac{dy}{dx}-y=e^xy^2,\;y(0)=1$ (note: $y=0$ does not satisfy $y(0)=1$)
\paragraph{Solution} From example 1: $\frac{1}{y}=\frac{-e^x}{2}+Ce^{-x}$
\\Set $x=0$ and $y=1\Rightarrow \frac{1}{1}=\frac{-e^0}{2}+Ce^{-0}\Rightarrow1=\frac{-1}{2}+C\Rightarrow C=\frac{3}{2}$
\\The solution is $\frac{1}{y}=\frac{-e^x}{2}+\frac{3}{2}e^{-x}$

\subsubsection{Example 3}
Solve the DEs using an appropriate substitution: $xy'-x^5y^{\frac{1}{3}}=3y$
\paragraph{Solution} $\Rightarrow y'-\frac{x^5}{x}y^{\frac{1}{3}}=\frac{3}{x}y$
\\$\Rightarrow y'-\frac{3}{x}y=x^4y^{\frac{1}{3}}\;[*]$ This is Bernoulli's equation with $n=\frac{1}{3}$
\\Let $u=y^{1-n}=y^{1-\frac{1}{3}}=y^{\frac{2}{3}}$
\\$u'-(\frac{2}{3})\frac{3}{x}u=\frac{2}{3}x^4$
\\$\frac{du}{dx}-\frac{2}{x}u=\frac{2}{3}x^4\;[2]$ linear equation in $u$
\\\\$I.F.=e^{\int \frac{-2}{x}dx}=e^{-2\ln|x|}=e^{\ln|x|{^{-2}}}=|x|^{-2}=\frac{1}{x^2}$
\\Multiply [2] by $\frac{1}{x^2}\Rightarrow \frac{1}{x^2}\frac{du}{dx}-\frac{2}{x}\frac{1}{x^2}u=\frac{2}{3}\frac{1}{x^2}$
\\$\int\frac{d}{dx}(\frac{1}{x^2}u)dx=\int\frac{2}{3}x^2dx$
\\$\frac{1}{x^2}u=\frac{2}{3}\frac{x^3}{3}+C\Rightarrow u=\frac{2}{9}x^3x^2+Cx^2$
\\Replace $u=y^{\frac{2}{3}}\Rightarrow y^{\frac{2}{3}}=\frac{2}{9}x^5+Cx^2$

\subsection{Linear Substition}
A DE of the form $\frac{dy}{dx}=f(Ax+By+C)$ where $B\neq0$ can be converted to a separable equation by using the substitution $u=Ax+By+C$, then solve the separable equation and replace $u$.

\subsubsection{Example 1}
Solve the DE by using an appropriate substitution: $\frac{dy}{dx}=2+e^{y-2x+6}$
\paragraph{Solution} Let $u=y-2x+6\Rightarrow y=u+2x-6$ and $\frac{dy}{dx}=\frac{du}{dx}+2$
\\Sub into original equation: $\frac{du}{dx}+2=2+e^u\Rightarrow\frac{du}{dx}=e^u\Rightarrow\int\frac{du}{e^u}=\int{dx}\Rightarrow$ separable equation.
\\$\frac{e^{-u}}{-1}=x+C$, replace $u=y+-2x+6\Rightarrow -e^{-(y-2x+6)}=x+C$

\subsubsection{Example 2}
Solve the DE by using an appropriate substitution: $\frac{dy}{dx}=\frac{1-x-y}{x+y}$
\paragraph{Solution} $\Rightarrow \frac{dy}{dx}=\frac{1-(x+y)}{x+y}$, we can have a linear substitution $f(x+y)$
\\Let $u=x+y\Rightarrow y=u-x$ and $\frac{dy}{dx}=\frac{du}{dx}-1$
\\Sub into original equation: $\frac{du}{dx}-1=\frac{1-u}{u}\Rightarrow\frac{du}{dx}=1+\frac{1-u}{u}=\frac{u+1-u}{u}=\frac{1}{u}$
\\$udu=dx$ (cross multiplication of $\frac{du}{dx}=\frac{1}{u}$), so $\int udu=\int dx\Rightarrow$ separable equation
\\$\frac{u^2}{2}=x+C$
\\Replace $u=x+y\Rightarrow\frac{(x+y)^2}{2}+x+C$
\\\\Now, back to the beginning: $\frac{Dy}{dx}=\frac{1-x-y}{x+y}$ is not separable, ont linear, not Bernoulli's equation.
\\We can check homogeneous and exact$\Rightarrow Mdx+Ndy=0\Rightarrow(x+y)dy=(1-x-y)dx
\\\Rightarrow(1-x-y)dx-(x+y)dy=0\Rightarrow$ not homogeneous ($1-x-y$ is not homogeneous)
\\$M=1-x-y$ and $N=-(x+y)\Rightarrow M_y=-1$ and $N_x=-1\Rightarrow$ exact equation.
\\You can solve by exact equation to get $x-\frac{x^2}{2}-xy-\frac{y^2}{2}=C$

\newpage\section{Real Life Problems}
Real life problems can be modelled as differential equations (especially first-order DEs). We will discuss this further:

\subsection{Linear Models} Real life applications can be written as 1st order linear DEs (section 2.1 in this document).
Let's look at some examples:

\subsubsection{Growth and Decay}
If $P$ is a population at time $t$, then the rate of change of that population is proportional to $P\Rightarrow\frac{dP}{dt}\propto P\Rightarrow \frac{dP}{dt}=kP$ where $k>0$
\\Radioactive elements decay with time. The rate of change of $A$, where $A$ represents the amount of decay,
is proportional to $A\Rightarrow\frac{dA}{dt}\propto A\Rightarrow\frac{dA}{dt}=kA$ where $k<0$

\subsubsection{Example 1}
The population of bacteria in a culture grows at a rate proportional to the number of bacteria present at time $t$.
After 3 hours, it is observed that 10,000 bacteria are present. If the initial population is 2000 bacteria,
when will the bacteria population reach 17,500?
\paragraph{Solution} The DE is $\frac{dP}{dt}=kP$ (where $P$ is the population) and the given conditions are $P(0)=2000$ and $P(3)=10,000$
\\To find $t$ where $P(t)=17,500$: $\frac{dP}{dt}=kP\Rightarrow\frac{dP}{dt}-kP=0$ and we now have a linear equation.
\\$I.F.=e^{\int -kdt}=e^{-kt}$
\\$e^{-kt}\frac{dP}{dt}-e^{-kt}kP=e^{-kt}(0)\Rightarrow\int\frac{d}{dt}(e^{-kt}P)dt=\int0dt$
\\\\Alternate solution: Make separable: $\int\frac{dP}{P}=\int kdt$ if $P\neq0$
\\$\ln|P|=kt+C\Rightarrow|P|=e^{kt+C}\Rightarrow P=e^{kt}\cdot e^C\Rightarrow P=C_1e^{kt}$
\\\\Anyways, $e^{-kt}P=C\Rightarrow P=\frac{C}{e^{-kt}}$
\\$P(0)=2000\Rightarrow2000=Ce^{k(0)}\Rightarrow2000=C\Rightarrow P=2000e^{kt}$
\\$P(3)=10,000\Rightarrow10000=2000e^{k{3}}\Rightarrow e^{3k}=\frac{10,000}{2000}\Rightarrow\ln{e^{3k}}=\ln5\Rightarrow3k=\ln5\Rightarrow k=\frac{\ln5}{3}\approx0.536$
\\So, $P=2000e^{0.536t}$
\\When $P=17,500\Rightarrow17,500=2000e^{0.536t}$
\\$e^{0.536t}=\frac{17,500}{2000}\Rightarrow\ln{e^{0.536t}}=\ln\frac{175}{20}\Rightarrow0.536t=\ln\frac{175}{20}\Rightarrow t=\frac{\ln\frac{175}{20}}{0.536}\approx4.044$
\\Therefore, the population will reach 17,500 bacteria a little after 4 hours.

\subsubsection{Example 2}
A sample of bismuth-210 decays at a rate proportional to the amount present at time $t$.
If 67\% of its original amount has decayed after 8 days, find the half life of this sample.
\paragraph{Solution} The DE is $\frac{dA}{dt}=kA$, where $A$ is the amount, given the conditions are $A(8)=\frac{33}{100}A_0$ where $A_0$ is the initial amount.
\\Reminder: Half-life refers to how long it takes for the substance to reduce to 50\% of its original amount.
\\From example 1: $A=Ce^{kt}$.
\\$A_0=Ce^{k(0)}=C(1)\Rightarrow C=A_0$
\\So $A=A_0e^{kt}$
\\Now, $A(8)=\frac{33}{100}A_0\Rightarrow\frac{33}{100}A_0=A_0e^{k(8)}$.
\\Take $\ln\Rightarrow\ln\frac{33}{100}=8k\Rightarrow k=\frac{\ln\frac{33}{100}}{8}\approx-0.139$
\\We thus have $A=A_0e^{-0.139t}$
\\When $A=\frac{1}{2}A_0\Rightarrow\frac{1}{2}A_0=A_0e^{-0.139t}\Rightarrow\ln\frac{1}{2}=-0.139t\Rightarrow t=\frac{\ln\frac{1}{2}}{-0.139}\approx4.985$

\newpage\subsubsection{Cooling and Warming} The DE is $\frac{dT}{dt}=k(T-T_m)$ where $T$ is the temperature of the object,
$t$ is time and $T_m$ is the temperature of the surroundings. Here, $k<0$.
\begin{itemize}
    \itemsep 0em
    \item When $T>T_m\Rightarrow\frac{dT}{dt}<0$ since $T-T_m>0$
    \item When $T<T_m\Rightarrow\frac{dT}{dt}>0$ since $T-T_m<0$
\end{itemize}

\subsubsection{Example 3 (from 3.1 \#14)}
A thermometer is taken from an inside room to the outside, where the air temperature is $5\dg$F.
After 1 minute, the termperature reads $55\dg$F and after 5 minutes, it reads $30\dg$F.
What is the initial temperature of the room?
\paragraph{Solution} The DE is $\frac{dT}{dt}=k(T-T_m)$ given $T(1)=55$ and $T(5)=30$
\\Since the air temperature is $5\dg$F, then $T_m=5\Rightarrow\frac{dT}{dt}=kT-5k$
\\$\frac{dT}{dt}-kT=-5k$
\\$I.F.=e^{\int-kdt}=e^{-kt}$
\\$e^{-kt}\frac{dT}{dt}-kTe^{-kt}=-5ke^{-kt}$
\\$\frac{d}{dt}(e^{-kt}T)=-5ke^{-kt}$
\\Integrating: $e^{-kt}T=\int-5ke^{-kt}dt=-5k\frac{e^{-kt}}{-k}+C=5e^{-kt}+C$
\\$T=\frac{5e^{-kt}}{e^{-kt}}+\frac{C}{e^{-kt}}\Rightarrow T=5+Ce^{kt}$
\\\\$T(1)=55\Rightarrow55=5+Ce^{k(1)}\Rightarrow Ce^k=50\;[1]$
\\$T(5)=30\Rightarrow30=5+Ce^{k(5)}\Rightarrow Ce^{5k}=25\;[2]$
\\$\frac{Ce^{5k}}{Ce^k}=\frac{25}{50}\Rightarrow e^{4k}=\frac{1}{2}\Rightarrow 4k=\ln\frac{1}{2}\Rightarrow k=\frac{\ln\frac{1}{2}}{4}\approx-0.173287$
\\Thus equation 1 becomes: $C=50e^{-k}=50e^{-(-0.173287)}\approx59.46$
\\So, $T=5+59.46e^{-0.173287t}$
\\At $t=0\Rightarrow T=5+59.46e^0\approx64.46$
\\Thus, the temperature of the room is $64.46\dg$F.


\subsubsection{Mixtures}
IF $A(t)$ is the amount of salt in a tank, then the DE is $\frac{dA}{dt}=R_{\text{in}}-R_{\text{out}}$
where $R_{\text{in}}$ is the input rate of salt and $R_{\text{out}}$ is the output rate of salt.
\\$R_{\text{in}}\Rightarrow$ (the input rate of flow) $\times$ (concentration of salt in solution).
\\$R_{\text{out}}\Rightarrow$ (the output rate of flow) $\times$ (concentration of salt).

\subsubsection{(3.1) Series Circuits (Section 1.3)}
We previously covered growth/decay, cooling/warming and mixtures. Now we're covering circuits:
\begin{itemize}
    \itemsep 0em
    \item For LR series, the DE is $L\frac{di}{dt}+Ri=E$, where $L$ is the inductance, $i$ is the current and $R$ is the resistance.
    \item For RC series, the DE is $R\frac{dq}{dt}+\frac{1}{c}q=F$, where $R$ is the resistance, $q$ is the charge and $c$ is the capacitance.
\end{itemize}

\subsubsection{Example}
A 200 volt electromotive force is applied to an RC series circuit in which the resistance is 100 ohms and the capacitance is $5\times 10^{-6}$ Farad.
Find the charge $q$ on the capacitor if $i(0)=0.4$. Determine the charge as $t\to\infty$. Find current at $t=0.005$ seconds.

\newpage\paragraph{Solution}
The DE is $R\frac{dq}{dt}+\frac{1}{c}q=E\Rightarrow 1000\frac{dq}{dt}+\frac{1}{5\times10^{-6}}q=200$
\\$\frac{dq}{dt}+\frac{10^6}{1000(5)}q=\frac{200}{1000}$
\\$\frac{dq}{dt}+200q=\frac{1}{5}$
\\$I.F.=e^{\int200dt}=e^{200t}$
\\\\$e^{200t}\frac{dq}{dt}+200e^{200t}q=\frac{1}{5}e^{200t}$
\\$\frac{d}{dt}(e^{200t}q)=\frac{1}{5}e^{200t}$
\\\\Integrating w.r.t. $t\Rightarrow e^{200t}q=\frac{1}{5}\frac{e^{200t}}{200}+C\Rightarrow q=\frac{e^{200t}}{1000e^{200t}}+\frac{C}{e^{200t}}$
\\$q=\frac{1}{1000}+Ce^{-200t}$
\\Given the initial condition $i(0)=0.4$, but $i=\frac{dq}{dt}=Ce^{-200t}(-200)$
\\$i(0)=0.4\Rightarrow0.4=Ce^{-200(0)}(-200)\Rightarrow C=\frac{0.4}{-200}=\frac{-1}{500}$
\\\textbf{Note:} the $0$ in $i(0)$ is the value of $t$ and the $0.4$ is the value of $i$.
\\The charge is $q=\frac{1}{1000}-\frac{1}{500}e^{-200t}$
\\\\$\Lim{t\to\infty}\left(\frac{1}{1000}-\frac{1}{500}e^{-200t}\right)=\frac{1}{1000}$
\\When $t\to\infty$, then the charge is $\frac{1}{1000}$ coulomb.
\\The current is $i=\frac{dq}{dt}=\frac{-1}{500}e^{-200t}(-200)$
\\$i=\frac{2}{5}e^{-200t}$
\\\\When $t=0.005$, $i=\frac{2}{5}e^{-200(0.005)}=\frac{2}{5}e^{-1}\approx0.1472$ amperes.

\section{(Ch. 4) Higher Order Differential Equations}

\subsection{Linear Differential Equations}
An nth order linear DE is $a_n(x)\frac{d^ny}{dx^n}+a_{n-1}(x)\frac{d^{n-1}y}{dx^{n-1}}+\dots+a_1(x)\frac{dy}{dx}+a_0(x)y=g(x)$
\\For IVPs: $y(x_0)=y_0, y'(x_0)=y_1,\dots,y^{(n-1)}(x_0)=y_{n-1}\Rightarrow$ you need $n$ conditions for an nth order DE.

\subsubsection{Example 1}
Given that $y=c_1+c_2\cos{x}+c_3\sin{x}$ is a solution of the DE $y'''+y'=0$ on the interval $(-\infty, \infty)$,
find the solution subject to the conditions: $y(\pi)=0$, $y'(\pi)=2$ and $y''(\pi)=-1$

\paragraph{Solution} $y(\pi)=0\Rightarrow0=c_1+c_2\cos\pi+c_3\sin\pi\Rightarrow c_1-c_2=0\;[1]$
\\$y'=c_2(-\sin{x})+c_3\cos{x}$
\\$y'(\pi)=2\Rightarrow2=c_2(-\sin\pi)+c_3\cos\pi\Rightarrow2=-c_3\Rightarrow c_3=-2$
\\$y''=c_2(-cos{x})+c_3(-\sin{x})$
\\$y''(\pi)=-1\Rightarrow-1=c_2(-\cos\pi)+c_3(-\sin\pi)\Rightarrow -1=c_2$
\\From [1], $c_1=c_2=-1$
\\Thus the solution is $y=-1-\cos{x}-2\sin{x}$

\subsection{Existence and Uniqueness Theorem for IVPs}
Consider the IVP: $a_n(x)y^{(n)}+a_{n-1}(x)y^{(n-1)}+\dots+a_1(x)y'+a_0(x)y=g(x)$ subject to $y(x_0)=y_0,\dots,y^{(n-1)}(x_0)=y_{n-1}$.
If $a_0(x)$, $a_1(x)$, $\dots$, $a_n(x)$ and $g(x)$ are continuous on an interval containing $x_0$ and $a_n(x)\neq0$ for any $x$ in the interval,
then the IVP has a unique solution.
\\\\Back in section 2.3 of the textbook, we covered first-order DEs had to be written in standard form: $a_1y'+a_0y=g(x)\Rightarrow y'+\frac{a_0}{a_1}y=\frac{g(x)}{a_1}$
where $\frac{a_0}{a_1}=P(x)$ and $\frac{g(x)}{a_1}=f(x)$, where $P(x)$ and $f(x)$ had to be continuous. This is a similar situation but for higher order DEs.

\newpage\subsubsection{Example}
Find an interval centered about $x=0$ for which the IVP has a unique solution: $(x-1)y''+(\sec x)y=e^x;\;y(0)=3,\:y'(0)=1$

\paragraph{Solution} $a_2=(x-1)$ and $e^x$ are continuous on $R$, while $a_0=\sec x=\frac{1}{\cos x}$ is continuous on $(\frac{-\pi}{2},\frac{\pi}{2})$
\\Thus the interval where the DE is continuous is $(\frac{-\pi}{2},\frac{\pi}{2})$
\\$a_2\neq0\Rightarrow x-1\neq0\Rightarrow x\neq1$
\\Because of that, the largest interval is $(-1,1)$

\subsubsection{Linear Dependence/Independence}
A set of functions $f_1,f_2,\dots,f_n$ are \textbf{linearly dependent} on I if there exists constants
$c_1,c_2,\dots,c_n$ where not all of them are 0 and $c_1f_1+c_2f_2+\dots+c_nf_n=0$
\\\\The \textbf{wronskian} of $f_1,f_2,\dots,f_n$ is $W=\begin{bmatrix}
    f_1 & f_2 & \dots & f_n \\
    f'_1 & f'_2 & \dots & f'_n \\
    f''_1 & f''_2 & \dots & f''_n \\
    \vdots & \ddots \\
    f^{(n-1)}_1 & \dots && f^{(n-1)}_n
\end{bmatrix}$
\\$f_1,f_2,\dots,f_n$ are linearly independent on an interval iff $W\neq0$ for all $x$ in $I$.

\subsubsection{Example 1} Determine wheter the function is linearly independent: $f_1(x)=e^x,f_2(x)=e^{-x}$ on the interval $(-\infty,\infty)$
\paragraph{Solution} $W=\begin{bmatrix}
    f_1 & f_2 \\
    f'_1 & f'_2
\end{bmatrix}=\begin{bmatrix}
    e^x & e^{-x} \\
    e^x & -e^{-x}
\end{bmatrix}=-e^xe^{-x}-e^{-x}e^x=-1-1=-2\neq0$
\\Thus the functions are linearly independent.

\subsection{Fundamental Set of Solutions}
$y_1,y_2,\dots,y_n$ are the fundamental set of solutions of an nth order DE on I if: \begin{enumerate}
    \item $y_1,y_2,\dots,y_n$ satisfy the DE on I, and
    \item $y_1,y_2,\dots,y_n$ are linearly independent (l.i.) on I.
\end{enumerate}

\subsubsection{Example}
Show that $y_1=\cos5x$ and $y_2=\sin5x$ are fundamental set of solutions of the DE $y''+25y=0$
\paragraph{Solution} $W=\begin{bmatrix}
    \cos5x & \sin5x \\
    (-\sin5x)(5) & 5\cos5x
\end{bmatrix}=5\cos^25x+5\sin^25x=5(\cos^25x+\sin^25x)=5\neq0\Rightarrow$ l.i.
\\$y=\cos5x,y'=-5\sin5x,y''=-25\cos5x\Rightarrow -25\cos5x+25\cos5x=0\Rightarrow0=0$
\\$y=\sin5x,y'=5\cos5x,y''=-25\sin5x\Rightarrow -25\sin5x+25\sin5x=0\Rightarrow0=0$
\\So $y_1,y_2$ are fundamental set of solutions.

\end{document}