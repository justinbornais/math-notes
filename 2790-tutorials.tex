\documentclass{article}
\usepackage{color}
\usepackage{amsmath}
\usepackage{graphicx}
\usepackage{geometry}
\usepackage{xcolor}
\geometry{letterpaper, total={170mm,257mm}, left=20mm, top=20mm}
\graphicspath{ {./images/} }
\newcommand{\Lim}[1]{\raisebox{0.5ex}{\scalebox{0.8}{$\displaystyle \lim_{#1}\;$}}}
\newcommand{\p}{\partial}
\newcommand{\dg}{^{\circ}}
\newcommand{\tri}{\vec{\bigtriangledown}}
\newcommand{\mat}[4]{\begin{bmatrix} #1 & #2 \\ #3 & #4 \end{bmatrix}}
\newcommand{\mathree}[9]{\begin{bmatrix} #1 & #2 & #3 \\ #4 & #5 & #6 \\ #7 & #8 & #9 \end{bmatrix}}
\newcommand{\der}[2]{\frac{d #1}{d #2}}
\newcommand{\pat}[2]{\frac{\partial #1}{\partial #2}}
\newenvironment{ind}[1]
 {\begin{list}{}
         {\setlength{\leftmargin}{#1}}
         \item[]
 }
 {\end{list}}

\title{MATH2790 - Tutorials}
\author{Justin Bornais}
\date{May 19, 2023}

\begin{document}
\maketitle
\section{(2.3) Linear 1st Order DE}
Reminder of standard form: $y'+P(x)y=f(x)$ where $P(x)$ and $f(x)$ are continuous. In many real life applications, $P(x)$ or $f(x)$ may be \textit{piecewise continuous}.

\paragraph{Piecewise Linear DEs} $P(x)$ or $f(x)$ is piecewise continuous on I.
\paragraph{Example} Solve the IVP: $\frac{dy}{dx}+y=f(x),\;y(0)=1$
where $f(x)=
\begin{cases}
    1 & 0\leq x\leq 1 \\
    -1 & x > 1
\end{cases}
$

\paragraph{Solution} For $0\leq x\leq 1$, the DE is $\frac{dy}{dx}+y=1
\\I.F.=e^{\int P(x)dx}=e^{\int 1dx}=e^x$
\\Multiply the equation by I.F: $e^xy'+e^xy=e^x\Rightarrow \frac{d}{dx}\left(e^xy\right)=e^x
\\\int \frac{d}{dx}(e^xy)dx=\int e^xdx
\\e^xy=e^x+C_1\Rightarrow y=\frac{e^x}{e^x}+\frac{C_1}{e^x}\Rightarrow y=1+C_1e^{-x},\;0\leq x\leq1
$
\\\\For $x>1$, the DE is $\frac{dy}{dx}+y=-1
\\I.F.=e^{\int P(x)dx}=e^x$ (it's the same as the above case since $P(x)$ remains unchanged)
\\Multiply the equation by I.F: $e^xy'+e^xy=-e^x\Rightarrow \frac{d}{dx}\left(e^xy\right)=-e^x
\\\int \frac{d}{dx}(e^xy)dx=\int -e^xdx
\\e^xy=-e^x+C_2\Rightarrow y=\frac{-e^x}{e^x}+\frac{C_2}{e^x}\Rightarrow y=-1+\frac{C_2}{e^x}, x>1$
\\\\Thus $y=\begin{cases}
    1+C_1e^{-x} & 0\leq x\leq1 \\
    -1+C_2e^{-x} & x>1
\end{cases}\;\;\;y(0)=1\Rightarrow1=1+C_1e^{-0}\Rightarrow1=1+C_1\Rightarrow C_1=0$
\\So $y=\begin{cases}
    1 & 0\leq x\leq1\\
    -1+C_2e^{-x} & x>1
\end{cases}\;\;\;$
For $y$ to be the solution, we need $y$ to be continuous at $x=1$.
\\$y(1)=\lim_{x \to 1^-}y=\lim+{x\to 1^+}
\\1=\lim_{x\to 1^-}1=\lim_{x\to 1^+}\left(-1+C_2e^{-x}\right)
\Rightarrow1=1=-1+C_2e^{-1}\Rightarrow2=C_2e^{-1}\Rightarrow C_2=2e$
\\Thus the solution of the IVP is $y=\begin{cases}
    1 & 0\leq x\leq1 \\
    -1+2e^{1-x} & x>1
\end{cases}$


\subsubsection{Example 1.1 \#19} Verify that $\ln{\frac{2x-1}{x-1}}=t$ is an implicit solution of the DE $\frac{dx}{dt}=(x-1)(2x-1)$

\paragraph{Solution} Differentiate each term w.r.t. $t$.
\\$\frac{1}{\frac{2x-1}{x-1}}\cdot\frac{(2x')(x-1)-(2x-1)x'}{(x-1)^2}=1\Rightarrow\frac{1}{2x-1}\cdot\frac{(2x')(x-1)-(2x-1)(x')}{x-1}=1
\\x'\left[2x-2-2x+1\right]=(2x-1)(x-1)\Rightarrow x'=(x-1)(1-2x)\rightarrow DE
$
\\Now we will find one explicit solution: $\ln{\frac{2x-1}{x-1}}=t
\Rightarrow\frac{2x-1}{x-1}=e^t\Rightarrow2x-1=e^t(x-1)\Rightarrow2x-1=xe^t-e^t
\Rightarrow2x-xe^t=-e^t+1\Rightarrow x=\frac{-e^t+1}{2-e^t}$

\newpage\section{Solutions by Homogeneous Substitutions}
Reminder: $Mdx+Ndy=0$ is a homogeneous equation if both $M$ and $N$ are homogeneous functions of the same degree (that is, $f(tx,ty)=t^af(x,y)$ with a degree of $a$).
To solve a homogeneous equation, choose one of the following:
\begin{itemize}
    \itemsep 0em
    \item Let $y=ux$ and $dy=udx+xdu\to$ separable equation in $x$ and $u\to$ solve and replace $u=\frac{y}{x}$.
    \item Let $x=vy$ and $dx=vdy+ydv\to$ separable equation in $v$ and $y\to$ solve and replace $v=\frac{x}{y}$.
\end{itemize}
Homoegeneous equations can have a singular solution.

\subsubsection{Example}
Solve the DE using appropriate substitution: $(x^4+2y^4)dx-xy^3dy=0$
\paragraph{Solution} Both $M$ and $N$ are homogeneous of degree 4, thus the given DE is homogeneous.
\\Let $y=ux$ and $dy=udx+xdu\Rightarrow(x^4+2u^4x^4)dx-xu^3x^3(udx+xdu)=0$
\\$x^4dx+2u^4x^4dx-u^4x^4dx-u^3x^4du=0
\\x^4dx+u^4x^4dx-u^3x^5du=0
\\x^4dx+u^4x^4dx=u^3x^4du
\\x^4dx(1+u^4)=u^3x^5du
\\\frac{x^4}{x^5}dx=\frac{u^3}{1+u^4}du
\\\int\frac{1}{x}dx=\int\frac{u^3}{1+u^4}du\to$separable if $x\neq0$ and if $1+u^4\neq0$.
\\$\ln|x|=\frac{1}{4}\ln|1+u^4|+C$
\\Replace $u=\frac{y}{x}$: $\ln|x|=\frac{1}{4}\ln|1+\frac{y^4}{x^4}|+C$
\\$\ln|x|=\frac{1}{4}\ln\left|\frac{x^4+y^4}{x^4}\right|+C$
\\$\ln|x|=\frac{1}{4}\ln|x^4+y^4|-\frac{1}{4}\ln{x^4}+C$
\\$\ln|x|=\frac{1}{4}\ln|x^4+y^4|-\ln|x^4|^{\frac{1}{4}}+C
\\2\ln|x|=\ln|x^4+y^4|^{\frac{1}{4}}+C
\\\ln|x|^2-\ln|x^4+y^4|^{\frac{1}{4}}=C
\\\ln\left(\frac{x^2}{|x^4+y^4|^{\frac{1}{4}}}\right)=C$

\newpage
\section{Midterm Review}
The midterm is on June 15th (11:30 AM - 12:50 PM) in the Education Building, room 1101.
It covers:\begin{itemize}
    \itemsep 0em
    \item Chapter 1 (1.1, 1.2) \begin{itemize}
        \item Find interval of existence of solution, or
        \item Find a region where an IVP has a unique solution (no solving the equation). Take $y'=f(x,y)$ and check that $f$ and $\frac{\p f}{\p y}$ are continuous on the region containing $(x_0,y_0)$
        \begin{itemize}
            \item An example: take $\sqrt{x^2-4}y'=x^5y^2\Rightarrow y'=\frac{x^5y^2}{x^2-4}\Rightarrow$ $y'=f$ is continuous when $x^2+4>0\Rightarrow x>2$ or $x<-2$
            \item Also $\frac{\p f}{\p y}=\frac{x^5(2y)}{\sqrt{x^2-4}}$, also continuous when $x>2$ or $x<-2$
            \item Can use region $\{(x,y)|-5\leq x\leq-2, -3\leq y\leq 3\}$
        \end{itemize}
    \end{itemize}
    \item Chapter 2 (2.2-2.5) \begin{itemize}
        \item You may be given separable equations: $\frac{dy}{dx}=g(x)\cdot h(y)$
        \item Might be given a linear equation: $y'+P(x)y=f(x)$
        \item If a question provides an implicit solution and says to find an explicit solution, then simplify this implicit solution to get explicit.
        \item \textbf{Singular solutions} $\to$ need to check for singular solutions, in the form $\int\frac{d(\text{dependent})}{\dots(=0)}$ \begin{itemize}
            \item $\frac{dy}{dx}=\frac{y-1}{x}\Rightarrow \int\frac{dy}{y-1}=\int\frac{dx}{x}$ since $x\neq0$ then $x$ is not a solution.
            \item If $y=1$ then substitue in original equation and check if it satisfies or not.
        \end{itemize}
    \end{itemize}
    \item Chapter 3 (3.1) (Models) \begin{itemize}
        \item Think about growth/decay, cooling/warming, mixture and circuits.
    \end{itemize}
    \item Chapter 4 (4.1 and 4.2) \begin{itemize}
        \item Think about $W=\mathree{f_1}{f_2}{f_3}{f'_1}{f'_2}{f'_3}{f''_1}{f''_2}{f''_3}$. If $W\neq0$ for all $x$ in $I\Rightarrow$ L.I. Otherwise, L.D.
        \item Think about reduction of order: $y_2=uy_1$ and homogeneous linear DEs $\Rightarrow u=\int\frac{e^{\int-P(x)dx}}{y^2_1}dx$
        \item Exact equations $Mdx+Ndy=0$ If $M_y=N_x$ then it is exact. \begin{itemize}
            \item To solve, $\frac{\p f}{\p x}=M$ and $\frac{\p f}{\p y}=N$ and the solution is $f(x,y)=C$
            \item The I.F. to make it exact is either $\mu=e^{\int\frac{M_y-N_x}{N}dx}$ with no $y$ or $\mu=e^{\int\frac{N_x-M_y}{M}dy}$ with no $x$.
        \end{itemize}
        \item Solve a DE by using appropriate substitution: \begin{itemize}
            \item Homogeneous DE: $Mdx+Ndy=0$ ($M$ and $N$ are homogeneous of the same degree) $\Rightarrow$ Let $y=ux,dy=udx+xdu$ or let $x=vy,dx=vdy+ydv$, and make them separable.
            \item Bernoulli's Equation: $y'+f(x)y=f(x)y^n$, $n\in R$, $n\neq0$ and $n\neq1\Rightarrow$ Let $u=y^{1-n}$, $u'+(1-n)P(x)u=(1-n)f(x)\Rightarrow$ a linear equation.
            \item Linear Substitution: $y'=f(Ax+By+C)\Rightarrow$ Let $u=Ax+By+C\Rightarrow$ separable equation.
        \end{itemize}
    \end{itemize}
\end{itemize}

\newpage\section{(4.4) Method of Undetermined Coefficients}
\begin{center}
    \begin{tabular}{||c c||} 
     \hline
     $g(x)$ & $y_p$ \\ [0.5ex] 
     \hline\hline
     $3$ & $A$ \\ 
     \hline
     $x^3-4$ & $Ax^3+Bx^2+Cx+D$ \\
     \hline
     $e^{2x}$ & $Ae^{2x}$ \\
     \hline
     $\cos3x$ & $A\cos3x+B\sin3x$ \\
     \hline
     $\sin3x$ & $A\cos3x+B\sin3x$ \\
     \hline
     $3+e^{2x}$ & $A+Be^{2x}$ \\
     \hline
     $xe^{3x}$ & $(Ax+B)e^{3x}$ \\
     \hline
     $x\sin5x$ & $(Ax+B)\cos5x+(Cx+D)\sin5x$ \\ [1ex] 
     \hline
    \end{tabular}
\end{center}
To solve a nonhomogeneous equation: \begin{enumerate}
    \itemsep 0em
    \item Solve associated homoegenous equation and get $y_c$ (4.3).
    \item Find $y_p$. \begin{itemize}
        \itemsep 0em
        \item You have $a_n,a_{n-1},\dots,a_1,a_0$ as constants.
        \item $g(x)$ is a constant, polynomial, $e^{kx}$, $\sin{kx}$, $\cos{kx}$, their sum, difference or product.
        \item $g(x)$ cannot be $\ln{x}$, $\tan{x}$, etc.
    \end{itemize}
    \item The solution of nonhomogeneous equation is $y=y_c+y_p$
\end{enumerate}

\subsubsection{Example 2 (Example 1 in Lecture Notes)}
Solve the DE: $y''-9y=2e^{3x}$
\paragraph{Solution} The homogeneous DE is $y''-9y=0$. The auxiliary equation is $m^2-9=0\Rightarrow m^2=9\Rightarrow m=\pm3$
\\Then $y_c=c_1e^{3x}+c_2e^{-x}$
\\Let $y_p=Ae^{3x}$, $y'_p=3Ae^{3x}$, $y''_p=9Ae^{3x}\Rightarrow 9Ae^{3x}-9Ae^{3x}=2e^{3x}\Rightarrow0=2e^{3x}$
\\\\\textbf{Case 1:} There is no duplication of $y_p$ with any term of $y_c$.
The $y_p$ will be a linear combination of the linearly independent functions generated by repeated differentiation of $g(x)$.
\\\textbf{Case 2:} If any term in $y_{p_i}$ duplicates with any term of $y_c$, then we multiply $y_{p_i}$
by $x^n$ where $n$ is the smallest power that removes the duplication.

\paragraph{Redo Example 2} $y_c=c_1e^{3x}+c_2e^{-3x}$
\\Let $y_p=(Ae^{3x})x=Axe^{3x}$
\\$y'_p=Ae^{3x}+Ax3e^{3x}=Ae^{3x}+3Axe^{3x}$\qquad$y''_p=3Ae^{3x}+3Ae^{3x}+3Ax3e^{3x}=6Ae^{3x}+9Axe^{3x}$
\\The solution is $y=y_c=+y_p-c_1e^{3x}+c_2e^{-3x}+\frac{1}{3}xe^{3x}$ (notice no $c$ terms for $y_p$)

\subsubsection{Example 3}
Solve $y'''+2y''=2x+5-e^{-2x}$
\paragraph{Solution} The associated homogeneous equation is $y'''+2y''=0$
\\The auxiliary equation is $m^3+2m^2=0\Rightarrow m^2(m+2)=0\Rightarrow m=0,0,-2$
\\So $y_c=c_1e^{0x}+c_2xe^{0x}+c^3e^{-2x}\Rightarrow y_c=c_1+c_2x+c_3e^{-2x}$
\\\\Let $y_p=Ax^3+Bx^2+Cxe^{-2x}$
\\$y'_p=3Ax^2+2Bx+Ce^{-2x}-2Cxe^{-2x}$
\\$y''_p=6Ax-2B-2Ce^{-2x}-2Ce^{-2x}-2Cx(-2e^{-2x})=6Ax+2B-4Ce^{-2x}+4Cxe^{-2x}$
\\$y'''_p=6A-4Ce^{-2x}(-2)+4Ce^{-2x}+4Cx(-2e^{-2x})=6A+12Ce^{-2x}-8Cxe^{-2x}$
\\\\Substitute into original DE: $6A+12Ce^{-2x}-8Cxe^{-2x}+12Ax+4B-8Ce^{-2x}+8Cxe^{-2x}=2x+5-e^{-2x}$
\\$\Rightarrow 6A+4Ce^{-2x}+12Ax+4B=2x+5-e^{-2x}$
\\\\Complete coefficients of $x$: $12A=2\Rightarrow A=\frac{2}{12}=\frac{1}{6}$
\newpage Constants: $6A+4B=5\Rightarrow 6(\frac{1}{6})+4B=5\Rightarrow4B=4\Rightarrow B=1$
\\$e^{-2x}:\;12C-8C=-1\Rightarrow 4C=-1\Rightarrow C=\frac{-1}{4}$
So $y_p=\frac{1}{6}x^3+x^2-\frac{1}{4}e^{-2x}$
\\\\Thus the solution is $y=y_c+y_p=c_1+c_2x+c_3e^{-2x}+\frac{1}{6}x^3+x^2-\frac{1}{4}e^{-2x}$

\subsubsection{Example 4}
Suppose the method of undetermined coefficients is to be used to solve the DE: $y''-10y'+25y=2e^{-5x}+xe^{5x}$ and $y_c=c_1e^{5x}+c_2e^{5x}$.
Write an appropriate form of $y_p$ (do not solve for coefficients).
\paragraph{Solution} $y_p=Ae^{-5x}+(Bx+C)e^{5x}x^2$ is proper form.
\\Note: $Bx^3+Cx^2$ was the original form, but $x^2$ was factored out.

\subsubsection{Example 5}
Suppose the DE $y^{(5)}+49y'''=x^2\cos{7x}+4x^2$ has to be solved by the method of undetermined coefficeints. Write the appropriateform of $y_p$ (do not solve for coefficients).
\paragraph{Solution} $y_p=\left[(Ax^2+Bx+C)\cos{7x}+(Dx^2+Ex+F)\sin{7x}\right]x+\left(\cos{x^2}+Hx+I\right)x^3$
\\Note: $y_{p_1}=(Ax^2+Bx+C)\cos{7x}+(Dx^2+Ex+F)\sin{7x}$ and $y_{p_2}=\cos{x^2}+Hx+I$
\\\\Now, for $y_c$: the auxiliary equation is $m^5+49m^3=0\Rightarrow m^3(m^2+49)=0$
\\Thus $m=0,0,0,\pm7i$
\\Thus $y_c=c_1+c_2x+c_3x^2+c_4\cos{7x}+c_5\sin{7x}$ (for $c_4$, $e^{0x}=1$).
\\\\Thus, $y_p=(Ax^3+Bx^2+Cx)\cos7x+(Dx^3+Ex^2+Fx)\sin7x+Gx^5+Hx^4+Ix^3$


\end{document}