\documentclass{article}
\usepackage{color}
\usepackage{amsmath}
\usepackage{graphicx}
\usepackage{geometry}
\usepackage{xcolor}
\geometry{letterpaper, total={170mm,257mm}, left=20mm, top=20mm}
\graphicspath{ {./images/} }
\newcommand{\Lim}[1]{\raisebox{0.5ex}{\scalebox{0.8}{$\displaystyle \lim_{#1}\;$}}}
\newcommand{\p}{\partial}
\newcommand{\dg}{^{\circ}}
\newcommand{\tri}{\vec{\bigtriangledown}}
\newcommand{\mat}[4]{\begin{bmatrix} #1 & #2 \\ #3 & #4 \end{bmatrix}}
\newcommand{\mathree}[9]{\begin{bmatrix} #1 & #2 & #3 \\ #4 & #5 & #6 \\ #7 & #8 & #9 \end{bmatrix}}
\newcommand{\der}[2]{\frac{d #1}{d #2}}
\newcommand{\pat}[2]{\frac{\partial #1}{\partial #2}}
\newenvironment{ind}[1]
 {\begin{list}{}
         {\setlength{\leftmargin}{#1}}
         \item[]
 }
 {\end{list}}

\title{MATH2780 - Tutorials}
\author{Justin Bornais}
\date{May 19, 2023}

\begin{document}
\maketitle
\section{(13.3) Arc Length \& Curvature}
Reminder of common formulas: \LARGE
\begin{align*}
    K&=\frac{|\vec{T}\,'(t)|}{|\vec{r}\,'(t)}  & K&=\frac{|\vec{r}\,'\times \vec{r}\,''|}{|\vec{r}\,'|^3} & \vec{T}(t)&=\frac{\vec{r}\,'(t)}{|\vec{r}\,'(t)|}
\end{align*}
\normalsize

\subsection{Unit Normal Vector}
The unit normal vectors tell us the direction in which the curve is turning. The vector $\vec{N}$ points towards the inside of the curve. The formula for the unit normal vector is as follows:

\LARGE$\vec{N}(t)=\frac{\vec{T}\,'(t)}{|\vec{T}\,'(t)}$\normalsize
\\The normal vector $\vec{N}$ is orthogonal to $\vec{T}$. Additionally, $\vec{T}$ and $\vec{T}\,'$ are orthogonal (or perpendicular). That means $\vec{N}\cdot\vec{T}=0$ and $\vec{T}\cdot\vec{T}\,'=0$.
\begin{itemize}
    \item Also note that $\vec{a}\cdot\vec{a}=|\vec{a}|^2$ or $|\vec{a}|\,|\vec{a}|\cos{0}=|\vec{a}|^2$
\end{itemize}

\subsubsection{Example 1}
Find the unit tangent vector and unit normal vector of the following curve: $\vec{r}(t)=<t,t^2>$
\paragraph{Solution} $\vec{T}(t)=\frac{\vec{r}\,'(t)}{|\vec{r}\,'(t)|}=\frac{<1,2t>}{\sqrt{1+4t^2}}=<\frac{1}{\sqrt{1+4t^2}},\frac{2t}{\sqrt{1+4t^2}}>$
\\$\vec{T}\,'(t)=<\frac{1}{2}(1+4t^2)^{\frac{-3}{2}}(8t),\frac{2}{1+4t^2}>$
\\\textbullet\ Side math: $\frac{d}{dx}\left(\frac{2t}{\sqrt{1+4t^2}}\right)
    =\frac{(2)\sqrt{1+4t^2}-2t\cdot\frac{1}{2}(1+4t^2)^{\frac{-1}{2}}(8t)}{\left(\sqrt{1+4t^2}\right)^2}
    =\frac{2\sqrt{1+4t^2}-\frac{8t^2}{\sqrt{1+4t^2}}}{1+4t^2}
    =\frac{\frac{2(1+4t^2)-8t^2}{\sqrt{1+4t^2}}}{1+4t^2}
    =\frac{2+8t^2-8t^2}{(1+4t^2)^{\frac{3}{2}}}$
\\$|\vec{T}\,'(t)|=\sqrt{\frac{16t^2}{(1+4t^2)^3}+\frac{4}{(1+4t^2)^3}}
    =\sqrt{\frac{16t^2+4}{(1+4t^2)^3}}=\sqrt{\frac{4(4t^2+1)}{(1+4t^2)^3}}
    =\sqrt{\frac{4}{(1+4t^2)^2}}=\frac{2}{1+4t^2}$
\\$\vec{N}(t)=\frac{\vec{T}\,'(t)}{|\vec{T}\,'(t)|}=\frac{<\frac{-4t}{(1+4t^2)^\frac{3}{2}},\frac{2}{(1+4t^2)^\frac{3}{2}}>}{\frac{2}{1+4t^2}}=<\frac{-2t}{\sqrt{1+4t^2}},\frac{1}{\sqrt{1+4t^2}}>$

\subsubsection{Example 2}
Find the unit tangent vector and unit normal vector of the following curve: $\vec{r}(t)=<3\cos{t},3\sin{t},2t>$
\paragraph{Solution} $\vec{r}\,'(t)=<-3\sin{t},3\cos{t},2>$
\\$|\vec{r}\,'(t)|=\sqrt{9\sin^2t+9\cos^2t+4}=\sqrt{13}$ (note that $9\sin^2t+9\cos^2t=9$)
\\$\vec{T}(t)=\frac{\vec{r}\,'(t)}{|\vec{r}\,'(t)|}=\frac{<-3\sin{t},3\cos{t},2>}{\sqrt{13}}=<\frac{-3}{\sqrt{13}}\sin{t},\frac{3}{\sqrt{13}}\cos{t},\frac{2}{\sqrt{13}}=\frac{1}{\sqrt{13}}<-3\sin{t},3\cos{t},2>$
\\$\vec{T}\,'(t)=\frac{1}{\sqrt{13}}<-3\cos{t},-3\sin{t},0>$
\\$|\vec{T}\,'(t)|=\sqrt{\frac{1}{13}\left(9\cos^2t+9\sin^2t\right)}=\sqrt{\frac{1}{13}(9)}=\frac{\sqrt{9}}{\sqrt{13}}=\frac{3}{\sqrt{13}}$
\\$\vec{N}(t)=\frac{\vec{T}\,'(t)}{|\vec{T}\,'(t)|}=\frac{\frac{1}{13}<-3\cos{t},-3\sin{t},0>}{\frac{3}{\sqrt{13}}}=\frac{<-3\cos{t},-3\sin{t},0>}{3}=<-\cos{t},-\sin{t},0>$

\newpage
\section{(14.2) Limits}
In last class, we discussed limits. Namely, for $\Lim{(x,y)\to(a,b)}f(x,y)$ if we get an answer, we get the limit.
If we get $\frac{0}{0}$ then we need to do something:
\begin{itemize}
    \itemsep 0em
    \item Factorization.
    \item Rationalization.
    \item Polar coordinates $\left(x^2+y^2=r^2,\;x=r\cos\theta,\;y=r\sin\theta,\;r\to0,\;\theta\in[0,2\pi]\right)$.
    \item Show two paths with different answers, so the limit DNE.
\end{itemize}
\subsubsection{Example 1}
Evaluate the limit if it exists, or show that the limit DNE: $\Lim{(x,y)\to(0,0)}\frac{x^2+y^4+y^2}{5x^2+5y^2}$
\paragraph{Solution} Use polar coordinates $x=r\cos\theta,\;y=r\sin\theta$
\\$\Lim{(x,y)\to(0,0)}\frac{x^2+y^4+y^2}{5x^2+5y^2}=\Lim{r\to0}\frac{r^2\cos^2\theta+r^4\sin^4\theta+r^2\sin^2\theta}{5r^2\cos^2\theta+5r^2\sin^2\theta}
=\Lim{r\to0}\frac{r^2+r^4\sin^4\theta}{5r^2(\cos^3\theta+\sin^2\theta)}=\Lim{r\to0}\frac{r^2(1+r^2\sin^4\theta)}{5r^2}
=\Lim{r\to0}\frac{1+r^2\sin^4\theta}{5}
\\=\frac{1+0}{5}=\frac{1}{5}$

\subsubsection{Example 2}
Evaluate the limit if it exists, or show that the limit DNE: $\Lim{(x,y)\to(0,0)}\frac{xy}{x^2+y^2}$
\paragraph{Solution} Along $x=0\Rightarrow\Lim{y\to0}\frac{0}{y^2}=\Lim{y\to0}0=0$
\\However, along $y=x\Rightarrow\Lim{x\to0}\frac{x(x)}{x^2+x^2}=\Lim{x\to0}\frac{x^2}{2x^2}=\frac{1}{2}$
\qquad Thus the limit DNE.
\paragraph{Alternate Solution} $x=r\cos\theta,\:y=r\sin\theta\Rightarrow=\Lim{r\to0}\frac{(r\cos\theta)(r\sin\theta)}{r^2}=\lim{r\to0}\cos\theta\sin\theta=\cos\theta\sin\theta$
\\If $\theta=0\Rightarrow\text{the limit is }0,\;$ If $\theta=\frac{\pi}{4}\Rightarrow\text{the limit is}\left(\frac{1}{\sqrt{2}}\right)\left(\frac{1}{\sqrt{2}}\right)=\frac{1}{2}$

\subsubsection{Example 3}
Evaluate the limit if it exists, or show that the limit DNE: $\Lim{(x,y)\to(0,0)}\frac{x^2\cos^{-1}\left(\frac{x}{\sqrt{x^2+y^2}}\right)}{3x^2+3y^2}$
\paragraph{Solution} Using the same polar coordinates: $=\Lim{r\to0}\frac{r^2\cos^2\theta\cos^{-1}\left(\cos\theta\right)}{3r^2}=\frac{\cos^2\theta(\theta)}{3}$
\\\textbf{Note:} $\frac{x}{\sqrt{x^2+y^2}}=\frac{r\cos\theta}{\sqrt{r^2\cos^2\theta+r^2\sin^2\theta}}=\frac{r\cos\theta}{\sqrt{r^2(\cos^2\theta+\sin^2\theta)}}=\frac{r\cos\theta}{\sqrt{r^2(1)}}=\frac{r\cos\theta}{r}=\cos\theta$
\\If $\theta=0\Rightarrow\text{the limit is }\frac{0}{3}=0$, but if $\theta=\pi\Rightarrow\text{the limit is}\frac{(\cos^2\pi)(\pi)}{3}=\frac{\pi}{3}$\qquad Thus the limit DNE.

\subsubsection{Example 4}
Evaluate the limit if it exists, or show that the limit DNE: $\Lim{(x,y)\to(0,0)}\frac{x^3\cos^{-1}\left(\frac{x}{\sqrt{x^2+y^2}}\right)}{3x^2+3y^2}$
\paragraph{Solution} Using the same polar coordinates: $=\Lim{r\to0}\frac{r^3\cos^3\theta\cos^{-1}\left(\cos\theta\right)}{3r^2}=0$

\subsubsection{Example 5}
Evaluate the limit if it exists, or show that the limit DNE: $\Lim{(x,y,z)\to(4,0,0)}\frac{10x^2+y^2+z^2}{x-yz}$
\paragraph{Solution} $\Lim{(x,y,z)\to(4,0,0)}\frac{10x^2+y^2+z^2}{x-yz}=\frac{10(4)^2+0+0}{4-0}=\frac{160}{4}=40$

\newpage\subsubsection{Example 6}
Evaluate the limit if it exists, or show that the limit DNE: $\Lim{(x,y,z)\to(1,0,2)}\frac{4x-2z}{2xy-zy+2x-z}$
\paragraph{Solution} $\Lim{(x,y,z)\to(1,0,2)}\frac{4x-2z}{2xy-zy+2x-z}=\frac{4(1)-2(2)}{2(1)(0)-2(0)-2(1)-2}=\textcolor{red}{\frac{0}{0}}$
\\Instead: $=\Lim{(x,y,z)\to(1,0,2)}\frac{2(2x-z)}{(2x-z)(y+1)}=\frac{2}{0+1}=2$

\subsubsection{Example 7}
Evaluate the limit if it exists, or show that the limit DNE: $\Lim{(x,y,z)\to(0,0,0)}\frac{6xz^2}{x^3+y^2+z^3}$
\paragraph{Solution} Along the $x$-axis (set $y=0$ and $z=0$) $=\Lim{x\to0}\frac{6x(0)^2}{x^3+0+0}=\Lim{x\to0}\frac{0}{x^3}=\Lim{x\to0}0=0$
\\Similarly, along the $y$ and $z$ axes respectively, you will get the same value of $0$.
\\However, along $y=0$, $z=x\Rightarrow\Lim{x\to0}\frac{6xx^2}{x^3+0+x^3}=\Lim{x\to0}\frac{6x^3}{2x^3}=3$
\\Since 2 paths give different limits, the limit DNE.

\section{(14.7) Maximum and Minimum Values}
In last class, $(a,b)$ is a critical point of $f$ if $f_x(a,b)=0$ and $f_y(a,b)=0$ or if one of $f_x$ and $f_y$ does not exist.
\\A critical point is a candidate for a maximum or minimum value. To check, we can use the \textbf{Second Derivative Test:}
Check if $(a,b)$ is a critical point by calculating $D=\begin{bmatrix}
    f_{xx} & f_{xy} \\
    f_{xy} & f_{yy}
\end{bmatrix}$
\begin{itemize}
    \itemsep 0em
    \item If $D(a,b)>0$ and $f_{xx}(a,b)<0$ then there is a local maximum at $(a,b)$.
    \item If $D(a,b)>0$ and $f_{xx}(a,b)>0$ then there is a local minimum at $(a,b)$.
    \item If $D(a,b)<0$ then $(a,b)$ is a saddle point.
\end{itemize}

\subsubsection{Example 1}
Find the critical points of the function $f(x,y)=2x^3+xy^2+5x^2+y^2$. Classify them as a local max, local min, or a saddle point.
\\The question could also be: Find the local max and min values of the function.

\paragraph{Solution} The critical points occur where $f_x=0$ and $f_y=0$
\\$f_x=6x^2+y^2+10x=0\;[1]$
\\$f_y=2xy+2y=\;[2]\Rightarrow 2y(x+1)=0\Rightarrow y=0$ or $x+1=0\Rightarrow x=-1$
\\\\If $y=0$ then equation [1] becomes $6x^2+10x=0\Rightarrow2x(3x+5)=0\Rightarrow x=0$ or $3x+5=0\Rightarrow x=\frac{-5}{3}$
\\Thus we have the critical points $(0,0)$ and $(\frac{-5}{3},0)$
\\\\If $x=-1$ then equation [1] becomes $6(-1)^2+y^2+10(-1)=0\Rightarrow y^2=4\Rightarrow y=\pm2$
\\Thus we also have the critical points $(-1,2)$ and $(-1,-2)$
\\\\$f_{xx}=6(2x)+0+10(1)=12x+10$
\\$f_{xy}=0+2y+0=2y$
\\$f_{yy}=2x(1)+2(1)=2x+2$
\\$D=\mat{f_{xx}}{f_{xy}}{f_{xy}}{f_{yy}}=\mat{12x+10}{2y}{2y}{2x+2}=(12x+10)(2x+2)-4y^2$
\\\\$D(0,0)=\mat{0+10}{0}{0}{0+2}=20-0=20>0\Rightarrow f_{xx}(0,0)=10>0\Rightarrow$ Local minimum occurs at $(0,0)$.
\\The local minim value is $f(0,0)=0$
\\$D(\frac{-5}{3},0)=\mat{12\frac{-5}{3}+10}{0}{0}{2\frac{-5}{3}+2}=\mat{-10}{0}{0}{\frac{-4}{3}}=(-10)(\frac{-4}{3})=\frac{40}{3}>0\Rightarrow f_{xx}(\frac{-5}{3},0)=-10<0$
\\Thus we have a local max at $(\frac{-5}{3},0)$. The local max value is:
\\$f(\frac{-5}{3},0)=2(\frac{-5}{3})^3+(\frac{-5}{3})(0)^2+5(\frac{-5}{3})^2+0^2=(\frac{-5}{3})^2(\frac{-10}{3}+5)=\frac{25}{9}(\frac{-10+15}{3})=\frac{25}{9}\cdot\frac{5}{3}=\frac{125}{27}$
\newpage$D(-1,2)=\mat{12(-1)+10}{2(2)}{2(2)}{2(-1)+2}=\mat{-2}{4}{4}{0}=0-16=-16<0\Rightarrow(-1,2)$ is a saddle point.
\\$D(-1,-2)=\mat{-2}{-4}{-4}{0}=0-16=-16<0\Rightarrow(-1,-2)$ is a saddle point.
\\This will be discussed in Chapter 14.7, which is optimization (word problems).

\section{Maximum and Minimum}
\subsubsection{Example 1}
Find the dimension of the rectangular box with the largest volume if the total surface area is 64cm$^2$.
\paragraph{Solution} We have to maximize $V=xyz\;[1]$
subject to $SA=2xy+2yz+2xz=64\Rightarrow xy+yz+xz=32\;[2]$
\\In last class, we used Lagrange multipliers to find the maximum of the volume ($f(x,y,z)$) subject to the surface area ($g(x,y,z)$).
\\\\Alternate method: From equation 2: $yz+xz=32-xy$
\\$z(y+x)=32-xy$
\\$z=\frac{32-xy}{y+x}$
\\Sub $z$ into equation 1: $V=xy\left(\frac{32-xy}{x+y}\right)$
\\We now have to maximize $f(x,y)=\frac{32xy-x^2y^2}{x+y}$, $x>0$, $y>0$, $z\geq0\Rightarrow xy\leq32$
\\Critical points: $f_x=0$ and $f_y=0$.
\\$f_x=\frac{(x+y)(32y-2xy^2)-(32xy-x^2y^2)(1)}{(x+y)^2}=\frac{32xy+32y^2-2x^2y^2-2xy^3-32xy+x^2y^2}{(x+y)^2}
=\frac{32y^2-x^2y^2-2xy^3}{(x+y)^2}=0
\\\Rightarrow y^2(32-x^2-2xy)=0\;[3]$
\\Similarly, $f_y=\frac{32x^2-y^2x^2-2yx^3}{(x+y)^2}=0\Rightarrow x^2(32-y^2-2xy)=0\;[4]$
\\\\From equation 3: $y^2=0$ or $32-x^2-xy=0$.
\\$y^2=0\Rightarrow y=0$ which is not in the domain.
\\So we have $32-x^2-xy=0\Rightarrow2xy=32-x^2\Rightarrow y=\frac{32-x^2}{2x}$
\\Substitute $y$ in equation 4: $x^2\left[32-\left(\frac{32-x^2}{2x}\right)^2-2x\left(\frac{32-x^2}{2x}\right)\right]=0$
\\We get $32-\frac{1024-64x^2+x^4}{4x^2}-(32-x^2)=0\Rightarrow32-\frac{1024-64x^2+x^4}{4x^2}-32+x^2=0$
\\$\Rightarrow-1024+64x^2-x^4+4x^4=0
\\\Rightarrow 3x^4+64x^2-1024=0$ Let $x^2=t$
\\$3t^2+64t-1024=0$
\\$t=\frac{-64\pm\sqrt{64^2-5(3)(-1024)}}{2(3)}=\frac{32}{3}$ or $\frac{-192}{6}$
\\Thus $x^2=\frac{32}{3}$ or $\frac{-192}{6}$, but $x>0$ so we only consider the first solution.
\\Thus $x=\sqrt{\frac{32}{3}}$. Do not consider the negative square root since $x>0$.
\\Then \Large$y=\frac{32-x^2}{2x}=\frac{32-\frac{32}{3}}{2\sqrt{\frac{32}{3}}}=\frac{2(\frac{32}{3})}{2\sqrt{32}{3}}=\sqrt{\frac{32}{3}}$
\\\normalsize Then \Large$z=\frac{32-\sqrt{\frac{32}{3}}\sqrt{\frac{32}{3}}}{\sqrt{\frac{32}{3}}+\sqrt{\frac{32}{3}}}=\frac{32-\frac{32}{3}}{2\sqrt{\frac{32}{3}}}
=\frac{2(\frac{32}{3})}{\sqrt{\frac{32}{3}}}=\sqrt{\frac{32}{3}}$\normalsize
\\\\Thus \Large$V=\frac{32\sqrt{\frac{32}{3}}\sqrt{\frac{32}{3}}-\left(\sqrt{\frac{32}{3}}\right)\left(\sqrt{\frac{32}{3}}\right)}{\sqrt{\frac{32}{3}}+\sqrt{\frac{32}{3}}}$
$=\frac{32\left(\frac{32}{3}\right)-\left(\frac{32}{3}\right)^2}{2\sqrt{\frac{32}{3}}}$
$=\frac{\frac{32}{3}\left(32-\frac{32}{3}\right)}{2\sqrt{\frac{32}{3}}}$
$=\frac{\frac{32}{3}\left(2\left(\frac{32}{3}\right)\right)}{2\sqrt{\frac{32}{3}}}$
$=\frac{32}{3}\sqrt{\frac{32}{3}}$\normalfont

\end{document}